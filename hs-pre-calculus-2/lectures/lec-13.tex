\lesson{13}{May 16 2022 Mon (20:01:59)}{Proving Trigonometric Identities}

Up until now, we've studied trigonometric identities that are useful. But there
are many other identities that aren't particularly important but they exist and
they offer us an opportunity to learn another skill, which is proving
mathematical statements.

Let's prove the identity $\sin (x) = \frac{\tan (x)}{\sec (x)}$. We prove
identities by manipulating the expression on one side of the equation until it
looks like the expression on the other side of the equation. We can start with
either side of the equation, but it's usually most common to start with the
\textbf{"more complicated"} side since it will be easier to manipulate it:

\begin{proof}
  \label{prf:sin_equals_tan_sec}

  \begin{align*}
    \frac{\tan (x)}{\sec (x)} &= \frac{\frac{\sin (x)}{\cos (x)}}{\frac{1}{\cos (x)}} \\
                              &= \frac{\sin (x)}{\cos (x)} \times \frac{\cos (x)}{1} \\
                     \sin (x) &= \sin (x) \\
  .\end{align*}
\end{proof}

Let's try to prove the identity $\cot (x) + \tan (x) = \csc (x) \sec(x)$. Here,
both sides are equally complicated so it's not obvious which side we should
start with. In such a case, just start with either side and see what happens.
If you get stuck, start over using the other side. Let's start with the left
side:

\begin{proof}
  \label{prf:cot_tan_equal_csc_sec}

  \begin{align*}
    \cot (x) + \tan (x) &= \frac{\cos (x)}{\sin (x)} + \frac{\sin (x)}{\cos (x)} \\
                        &= \frac{\cos (x)}{\sin (x)} \times \frac{\cos (x)}{\cos (x)} + \frac{\sin (x)}{\cos (x)} \times \frac{\sin (x)}{\sin (x)} \\
                        &= \frac{\cos^{2} (x) + \sin^{2} (x)}{\sin (x)\cos (x)} \\
                        &= \frac{1}{\sin (x)\cos (x)} \\
       \csc (x)\sec (x) &= \csc (x)\sec (x) \\
  .\end{align*}
\end{proof}

Let's try to prove the identity
$\frac{(1 + \cos (t))(1 - \cos (t))}{\sin (t)} = \sin (t)$. The left side is
more complicated, so I'll start with that:

\begin{proof}
  \label{prf:1_cos_1_cos_over_sin_equals_sin}

  \begin{align*}
    \frac{(1 + \cos (t))(1 - \cos (t))}{\sin (t)} &= \frac{1 - \cos^{2} (t)}{\sin (t)} \\
                                                  &= \frac{\sin^{2} (t)}{\sin (t)} \\
                                         \sin (t) &= \sin (t) \\
  .\end{align*}
\end{proof}

Let's try to prove the identity function $\frac{\cos (\theta)}{1 - \sin
(\theta)} = \frac{1 + \sin (\theta)}{\cos (\theta)}$. To prove identities like
this, we use a \textit{"trick"} called \textbf{conjugate}. A conjugate of an
expression is the opposite. For example, the conjugate of $a + b$ is $a - b$.

\begin{proof}
  \label{prf:cos_1_sin_equals_1_sin_cos}

  \begin{align*}
    \frac{\cos (\theta)}{1 - \sin (\theta)} &= \frac{\cos (\theta)}{1 - \sin (\theta)} \times \frac{1 + \sin (\theta)}{1 + \sin (\theta)} \\
                                            &= \frac{\cos (\theta)(1 + \sin (\theta))}{(1 - \sin (\theta))(1 + \sin (\theta))} \\
                                            &= \frac{\cos (\theta)(1 + \sin (\theta))}{1 - \sin^{2} (\theta)} \\
                                            &= \frac{\cos (\theta)(1 + \sin (\theta))}{\cos^{2} (\theta)} \\
    \frac{1 + \sin (\theta)}{\cos (\theta)} &= \frac{1 + \sin (\theta)}{\cos (\theta)}
  .\end{align*}
\end{proof}

\newpage
