\lesson{19}{Jun 01 2022 Wed (10:33:28)}{Intro to Vectors}
\label{les_19:intro_to_vectors}

Vectors are mathematical objects used to represent physical quantities like
velocity, force, and displacement.

\begin{definition}[Vectors]
  \label{def:vectors}

  A \textbf{vector} is a mathematical object that has both a magnitude and a
  direction.
\end{definition}

In order to distinguish between vectors and scalers, we need to use a different
notation to denote vectors.

\begin{notation}
  To denote a vector, we use a small arrow on top.
  \[ \overrightarrow{\vec{v}} \].
\end{notation}

A two-dimensional vector can be represented by an arrow on the coordinate
plane. The length of the arrow represents the \textbf{magnitude} of the vector
and the \textbf{direction} of the arrow represents the direction of the vector.

\begin{example}
  \label{exm:vector}

  The vector $\overrightarrow{\vec{v}}$ is depicted as an arrow on the
  coordinate plane.

  TODO: Draw vector on coordinate plane (with tikz)

  The \textbf{tip} of the vector is where the arrow ends and the \textbf{tail}
  of the vector is where the arrow begins. Therefore, the tip of
  $\overrightarrow{\vec{v}}$ is at the point $(4,3)$ and the tail of the vector
  is at the origin, $(0,0)$.
\end{example}

\begin{notation}
  We denote the magnitude of vector $\overrightarrow{\vec{v}}$ by
  $\Vert \overrightarrow{\vec{v}} \Vert$.
\end{notation}

To find the magnitude of $\overrightarrow{\vec{v}}$, we need to find the length
of the arrow. We an do this by thinking of the arrow as being the hypotenuse of
a right-triangle with side lengths of $4$ and $3$ and then use the Pythagorean
Theorem to find $\Vert \overrightarrow{\vec{v}} \Vert$.

\begin{align*}
  \Vert \overrightarrow{\vec{v}} \Vert &= \sqrt{a^{2} + b^{2}} \\
                                       &= \sqrt{4^{2} + 3^{2}} \\
                                       &= \sqrt{16 + 9} \\
                                       &= 5
.\end{align*}

We can find the angle between the positive $x$-axis and the arrow to describe
the \textbf{direction} of the vector. We can use trig to find the angle
$\theta$.

\begin{align*}
  \tan (\theta) &= \frac{3}{4} \\
         \theta &= \tan^{-1} \left(\frac{3}{4}\right) \\
                &\approx 36.9^{\circ} \\
.\end{align*}

Although the magnitude and direction of the vector describe it completely, it
is often useful to describe it by using its \textbf{horizontal} and
\textbf{vertical} components. An example is:
\[ \overrightarrow{\vec{v}} = \braket{4,3} \].

When graphing vectors, you don't need to start anywhere specific. For example,
all the arrows in Figure \ref{fig:translation_of_vector_anywhere} represent
$\overrightarrow{\vec{v}}$ since all of these vectors have a horizontal
component of $4$ units and a vertical component of $3$ units.

\subsection*{Vector Operations}
\label{sub_sec:vector_operations}

We can multiply any vector by a scalar and we can add or subtract any two
vectors.

When we multiply a vector by a scalar, we multiply the components of the vector
by the scalar. Thus, $\overrightarrow{\vec{a}} = \braket{a_1,a_2}$ and $k \in
\mathbb{R}$, then $k\overrightarrow{\vec{a}} = \braket{ka_1,ka_2}$ 

\begin{theorem}
  If $\overrightarrow{\vec{a}} = \braket{a_1,a_3}$ is a vector and $k \in
  \mathbb{R}$, then $k\overrightarrow{\vec{a}} = \braket{ka_1,ka_2}$ has a
  magnitude $\vert k \vert \times \Vert \overrightarrow{\vec{a}} \Vert$.
  If $k > 0$, then $k \times \overrightarrow{\vec{a}}$ points in the same
  direction as $\overrightarrow{\vec{a}}$. If $k < 0$, then $k \times
  \overrightarrow{\vec{a}}$ points in the opposite direction as
  $\overrightarrow{\vec{a}}$.
\end{theorem}

\begin{exc}[Solution \ref{sol:multiply_vectors_by_scalers}]
  \label{exc:multiply_vectors_by_scalers}

  Let $\overrightarrow{\vec{v}} = \braket{4,3}$. Find and draw vectors
  $\overrightarrow{\vec{m}} = 2 \overrightarrow{\vec{v}}$ and
  $\overrightarrow{\vec{n}} = -2\overrightarrow{\vec{v}}$:
\end{exc}

% subsection vector_operations (end)

\newpage
