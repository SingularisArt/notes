\lesson{9}{Apr 28 2022 Thu (09:01:09)}{Inverse Trig Functions}
\label{les_9:inverse_trig_functions}

For example, if $f$ and $f^{-1}$ are inverses of one another and if $f(a) = b$,
then $f^{-1}(b) = a$. Inverse functions are extremely valuable since they "undo"
one another and allow us to solve equations. For example, we can solve the
equation $x^{3} = 10$ by using the inverse of cubing function, namely the
cube-root of the function, to "undo" the cubing involved in the equation:
\begin{align*}
  x^{3} &= 10 \\
  \sqrt[3]{x^{3}} &= \sqrt[3]{10} \\
  x &= \sqrt[3]{10} \\
.\end{align*}

The cubing function has an inverse function because each output value
corresponds to exactly one input value. This means that the cubing function is
\textbf{one-to-one}, and it's only one-to-one functions whose inverses are also
functions.

Unfortunately, the trig functions aren't one-to-one so, in their natural form,
they don't have inverse functions. For example, consider the output
$\frac{1}{2}$ for the sin function. This output corresponds to the inputs,
$\frac{\pi}{6}, \frac{5\pi}{6}, \frac{13\pi}{6}, \frac{17\pi}{6}$, etc.

Since inverse functions can be so valuable, we really want inverse trig
functions, so we need to restrict the domains of the functions to intervals on
which they are on-to-one, and then we can construct inverse functions. Let's
start by constructing the inverse of the sin function.

In order to construct the inverse of the sin function, we need to restrict the
domain to an interval on which the function is one-to-one, and we need to choose
an interval of the domain that utilizes the entire range of the sin function,
$[-1,1]$. We'll choose the interval $-\frac{\pi}{2},\frac{\pi}{2}$.

\begin{definition}[Inverse Sin Function]
  \label{def:inverse_sin_function}

  The inverse sin function, denoted $y = \sin^{-1} (t)$, is defined by the
  following:
  \[
  \textrm{if } -\frac{\pi}{2} \le y \le \frac{\pi}{2} \textrm{ and } \sin(y) = t
  \textrm{then } y = \sin^{-1} (t)
  \].

  By construction, the range of $y = \sin^{-1} (t)$ is $[-\frac{\pi}{2},
  \frac{\pi}{2}]$, and the domain is the same as the range of the sin function:
  $[-1,1]$.

  \begin{note}
    The inverse sin function is often called the \textbf{arcsine function} and
    denoted $y = \arcsin(t)$
  \end{note}
\end{definition}

Now let's construct the inverse of the cos function. Like the sin function, the
cos function isn't one-to-one, s we'll need to restrict its domain to construct
the inverse cos function. The cos function is one-to-one on the interval $[0,
\pi]$ and, on this interval, the graph utilizes the entire range of the cos
function, $[-1,1]$. So we can define the inverse cos function on this portion of
the cos function.

\begin{definition}[Inverse Cos Function]
  \label{def:inverse_cos_function}

  The inverse cos function, denoted $y = \cos^{-1} (t)$, is defined by the
  following:
  \[
  \textrm{if } 0 \le y \le \pi \textrm{ and } \cos(y) = t \textrm{ then } y =
  \cos^{-1} (t)
  \].

  By construction, the range of $y = \cos^{-1} (t)$ is $[0, \pi]$, and the
  domain is the same as the range of the cos function: $[-1, 1]$.

  \begin{note}
    The inverse cos function is often called the \textbf{arccosine function} and
    denoted $y = \arccos (t)$
  \end{note}
\end{definition}

Now let's define the inverse tan function. The tan function is one-to-one on the
interval $(-\frac{\pi}{2},\frac{\pi}{2})$, since the period of tangent is $\pi$
units, this interval represents a complete period of tangent. In order to
construct the inverse tan function, we restrict the tan function to the interval
$(-\frac{\pi}{2},\frac{\pi}{2})$.

\begin{definition}[Inverse Tan Function]
  \label{def:inverse_tan_function}

  The inverse tan function, denoted $y = \tan^{-1} (t)$ is defined by the
  following:
  \[
  \textrm{if } -\frac{\pi}{2} \le y \le \frac{\pi}{2} \textrm{ and } \tan(y) = t
  \textrm{ then } y = \tan^{-1} (t)
  \].

  By construction, the range of $y = \tan^{-1} (t)$ is
  $(-\frac{\pi}{2},\frac{\pi}{2})$, and the domain is the same as the range of
  the tan function: $\mathbb{R}$.

  \begin{note}
    The inverse tan function is often called the \textbf{arctangent function}
    and denoted $y = \arctan (t)$
  \end{note}
\end{definition}

\begin{exc}[Solution \ref{sol:evaluate_invs_sin}]
  Evaluate the following expressions:

  \label{exc:evaluate_invs_sin}

  \begin{itemize}
    \label{item:evaluate_invs_sin}

    \item $\sin^{-1} (-\frac{1}{2})$
    \item $\sin (\sin^{-1} (\frac{\sqrt{3}}{2}))$
  \end{itemize}
\end{exc}

\begin{exc}[Solution \ref{sol:evaluate_invs_cos}]
  Evaluate the following expressions:

  \label{exc:evaluate_invs_cos}

  \begin{itemize}
    \label{item:evaluate_invs_cos}

    \item $\cos^{-1} (0)$
    \item $\cos (\cos^{-1} (\frac{1}{2}))$
  \end{itemize}
\end{exc}

\begin{exc}[Solution \ref{sol:evaluate_invs_tan}]
  Evaluate the following expressions:

  \label{exc:evaluate_invs_tan}

  \begin{itemize}
    \label{item:evaluate_invs_tan}

    \item $\tan^{-1} (1)$
    \item $\tan (\tan^{-1} (-\sqrt{3}))$
  \end{itemize}
\end{exc}

\newpage
