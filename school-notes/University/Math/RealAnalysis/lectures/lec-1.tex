\lesson{1}{Feb 20 2022 Sun (10:16:58)}{Introduction and Proofs}

\subsubsection{Logic}
\label{sub_sub_sec:logic}

Let's say we have proposition $P$ and $Q$. They can be joined by "and", "or",
"implies", or "if and only if".

For example, if $P$ is a proposition, then "not $P$" is a new propostion that
is true whenever $P$ is false and vice versa. The symbolic representation for
"not $P$" is $\lnot P$ or $\overline{P}$ 

Two propositions, $P$ and $Q$, can be joined by "and", "or", "implies", or "if
and only if" to form a new proposition. The truth of this new proposition is
determined by the truth of $P$ and $Q$ according to the \textbf{Truth Table}.

\begin{table}[htbp]
  \centering
  \begin{tabular}{cc|cccc}
    & & & & "$P$ implies $Q$ or" & "$P$ if and only if $Q$" or \\
    & & "$P$ and $Q$" & "$P$ or $Q$" & "if $P$, then $Q$" & "$P$ iff $Q$" \\
    $P$ & $Q$ & ($P \land Q$) & ($P \implies Q$) & ($P \implies Q$) & ($P \iff Q$) \\
    \hline
    F & F & F & F & T & T \\
    F & T & F & T & T & F \\
    T & F & F & T & F & F \\
    T & T & T & T & T & T \\
  \end{tabular}
  \label{tab:truth-table}
\end{table}

Here are a few hidden features within this table:

\begin{itemize}
  \item The phrase "$P$ or $Q$" is true if $P$ is true, $Q$, or both are true.
  \item The phrase "$P$ implies $Q$" is true when $P$ is false or $Q$ is true.
\end{itemize}

There are two more important phrases in mathematical writing: "for all"
(symbolized by $\forall$) and "there exists" (symbolized by $\exists$). These
are called \textbf{\textit{quantifiers}}.

\begin{definition}[Quantifiers]
  A quantifier is always followed by a variable (and perhaps an indication of
  the range of that variable) and then a predicate, which typically involves
  that variable. Here are a couple of examples:

  \begin{align}
    \forall x \in \mathbb{R}^{+} \quad e^{x} < (1 + x)^{1 + x} \\
    \exists n \in \mathbb{N} \quad 2^{n} > (100n)^{100}
  .\end{align}

  The first statement says that $e^{x}$ is less than $(1 + x)^{1 + x}$ for every
  positive real number $x$. The second statement says that there exists a
  natural number $n$ such that $2^{n} > (100n)^{100}$.

  The special symbols such as $\forall, \exists, \lnot,$ and $\land$ are useful
  to logicians who are trying to express mathematicians ideas without resorting to
  the English language at all. Also, other mathematicians use these symbols as
  shorthands.
\end{definition}

% subsubsection logic (end)

\subsubsection{Proving an Implication}
\label{sub_sub_sec:proving_an_implication}

Let's try to prove the following theorem:

\begin{theorem}[Let $P(a,b)$ be any predicate defined for all
  \label{thrm:let_p_a_b_be_any_predicate_defined_for_all}
  $a \in \mathbb{A}$ and $b \in \mathbb(B)$ Then:]

  \begin{equation}
    \begin{split}
      (\exists a \in \mathbb{A} \quad
      \forall b \in \mathbb{B} \quad
      P(a,b)) \quad
      \implies \quad
      (\forall b \in \mathbb{B} \quad
      \exists a \in \mathbb{A} \quad
      P(a,b))
    \end{split}
  .\end{equation}

  Let's impose a specific interpretation in order to give concrete meaning:

  \begin{equation}
    \begin{split}
      \mathbb{A} = \{6.042 \quad \textrm{students}\} \\
      \mathbb{B} = \{6.042 \quad \textrm{lectures}\} \\
      P(a,b) = \textrm{"student $a$ falls asleep during lecture $b$"}
    \end{split}
  .\end{equation}

  Interpreting the left side in these terms give:

  \begin{equation}
    \begin{split}
      \exists a \in \mathbb{A} \quad \forall b \in \mathbb{B} \quad P(a,b) \\
      = \textrm{"there exists a student that falls asleep in every lecture"}
    \end{split}
  .\end{equation}

  So, this side states that some particular student always falls asleep.
  Let's call him Snoozer. Now, here's the right side:

  \begin{equation}
    \begin{split}
      \forall b \in \mathbb{B} \quad \exists a \in \mathbb{A} \quad P(a,b) \\
      = \textrm{"in every lecture, some student falls asleep"}
    \end{split}
  .\end{equation}

  This is slightly different than the left side because there might be a
  different sleeper in each lecture. The left side should imply the right. If
  Snoozer sleeps in every lecture, then in every lecture some student is surely
  asleep.
\end{theorem}

% subsubsection proving_an_implication (end)

\newpage
