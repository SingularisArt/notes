\lesson{8}{Apr 23 2022 Sat (19:15:57)}{Sinusoidal Functions}
\label{les_8:sinusoidal_functions}

\begin{definition}[Sinusoidal Function]
  \label{def:sinusoidal_function}

  A \textbf{Sinusoidal Function} is a function of the form:
  \[ y = A\sin(\omega(t - h)) + k \textrm{ or } y = A\cos(\omega(t - h)) + k \].

  where $A, \omega, h, k \in \mathbb{R}$
\end{definition}

Based on what we know about graph transformations, we should recognize that a
sinusoidal function is a transformation of $y = \sin(t)$ or $y = \cos(t)$.
Consequently, sinusoidal function are waves with the same curvy shape as the
graphs of sin and cos but with different periods, midlines, and/or amplitudes.

\subsubsection*{Summary of Graph Transformation}
\label{sub_sub_sec:summary_of_graph_transformation}

Suppose that $f$ and $g$ are functions such that
$g(t) = A \times f(\omega(t - h)) = k$ and $A, \omega, h, k \in \mathbb{R}$. In
order to transform the graph of the function $f$ into the graph of $g$ \ldots

\begin{enumerate}
  \label{enum:steps_to_transform_f_to_g}

  \item horizontally stretch/compress the graph of $f$ by a factor of
    $\frac{1}{\mid \omega \mid}$ and, if $\omega < 0$, reflect it about the
    $y$-axis.
  \item shift the graph horizontally $h$ units (shift right if $h$ is positive
    and left if $h$ is negative).
  \item vertically stretch/compress the graph by a factor of $\mid A \mid$ and,
    if $A < 0$, reflect it about the $t$-axis.
  \item shift the graph vertically $k$ units (shift up if $k$ is positive and
    down if $k$ is negative).
\end{enumerate}

\begin{note}
  The order in which these transformations are performed matters.
\end{note}

% subsubsection summary_of_graph_transformation (end)

\begin{exc}[Solution \ref{sol:transform_the_graph_of_f_to_g}]
  \label{exc:transform_the_graph_of_f_to_g}

  Describe how we can transform the graph of $f(t) = \sin(t)$ into the graph of
  $g(t) = 2\sin(t) - 3$. State the period, midline, and amplitude of $y = g(t)$.
\end{exc}

Here's a quick summary of sinusoidal functions:

\begin{theorem}
  The graphs of the sinusoidal functions

  \[ y = A\sin(\omega(t - h)) + k \textrm{ and } y = A\cos(\omega(t - h)) + k \].

  where $A, \omega, h, k \in \mathbb{R}$ have the following properties:

  \begin{center}
    \textbf{period:} $\frac{2\pi}{\mid \omega \mid} \textrm{ units}$ \\
    \textbf{midline:} $y = k$ \\
    \textbf{amplitude:} $\mid A \mid$ units \\
    \textbf{horizontal shift:} $h$ units \\
    \textbf{angular frequency:} $\omega$ radians per unit of $t$
  \end{center}
\end{theorem}

Now let's try to sketch the graph of:
\[
f(t) = 2\sin \left(\pi t - \frac{\pi}{4}\right) - 3
\].

To do this, we need to find the midline, amplitude, horizontal shift, and
period. We can determine the midline by noticing that this function has a $-3$,
which causes the sin wave to be shifted down $3$ units.
The amplitude can be determined by the number outside the function, which will
cause a vertical stretch. So, our amplitude is $2$ units. To find the horizontal
shift, we first need to factor out $\pi$ from the function's input.
\[
f(t) = 2\sin \left(\pi \left(t - \frac{1}{4}\right)\right) - 3
\].
The reason we did this is because we need to find out how much we're moving left
or right, and $-\frac{1}{4}$ indicates that we're moving $\frac{1}{4}$ units to
the right.
The period comes from two features of the formula. The first one is we're using
the trig function sin, which has a period of $2\pi$ units and the second feature
is we have $\pi$ multiplying all of the numbers in the input, which horizontally
compresses the function.
\[
2\pi \times \frac{1}{\pi} = 2 \textrm{ units}
\].

Here's the final graph:

\begin{figure}[htbp]
  \centering

  \begin{tikzpicture}
    \begin{axis}[
        my axis style,
        width=\textwidth,
        height=.5\textwidth,
      ]
      \addplot[
        domain=-10:10,
        blue,
        thick,
        <->
      ]
      {2*sin(pi*(deg(x)-1/4))-3};
    \end{axis}
  \end{tikzpicture}

  \caption{}
  \label{fig:final_graph_for_exercise}
\end{figure}

\begin{exc}[Solution \ref{sol:find_two_algebraic_rules_for_the_function_y_g_t}]
  \label{exc:find_two_algebraic_rules_for_the_function_y_g_t}

  Find two different algebraic rules (one involving sin and one involving cos)
  for the function $y = g(t)$ graphed below:

  \begin{figure}[H]
    \centering

    \begin{tikzpicture}
      \begin{axis}[
          my axis style,
          width=\textwidth,
          height=.5\textwidth,
        ]
        \addplot[
          domain=-10:10,
          blue,
          thick,
          <->
        ]
        {4*sin(pi/3*(deg(x)+1))-3};
      \end{axis}
    \end{tikzpicture}

    \caption{}
    \label{fig:algebraic_rules}
  \end{figure}
\end{exc}

\newpage
