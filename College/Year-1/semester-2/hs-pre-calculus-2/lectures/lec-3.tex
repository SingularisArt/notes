\lesson{3}{Apr 07 2022 Thu (18:43:36)}{Introduction to Periodic Functions}
\label{les_3:introduction_to_periodic_functions}

Any activity that repeats on a regular time interval can be described as
\textit{periodic}.

\begin{definition}[Periodic Function]
  \label{def:periodic_function}

  A \textbf{periodic function} whose values repeat on regular intervals. Hence,
  $f$ is a periodic if there exists some constant $c$ such that:
  \[ f(x + c) = f(x) \].

  for all $x$ in the domain of $f$ such that $f(x + c)$ is defined.

  Recall that this means that if the graph $y = f(x)$ is shifted horizontally $c$ 
  units then it will appear unaffected.
\end{definition}

\begin{definition}[Period]
  \label{def:period}

  The \textbf{period} of a function $f$ is the smallest value $\mid c \mid$ such
  that $f(x + t)$ for all $x$ in the domain of $f$ such that $f(x + c)$ is
  defined.
\end{definition}

\begin{exc}[Solution \ref{sol:period}]
  \label{exc:period}

  Find the period of the function graphed below:

  \begin{figure}[H]
    \centering

    \begin{tikzpicture}
      \begin{axis}[
          my axis style,
          width=\textwidth,
          height=.5\textwidth,
        ]
        \addplot[
          domain=-10:10,
          red,
          thick,
          <->
        ]
        {sin(deg(x*3.14))};
      \end{axis}
    \end{tikzpicture}

    \caption{}
    \label{fig:period}
  \end{figure}
\end{exc}

\begin{definition}[Midline]
  \label{def:midline}

  The \textbf{midline} of a periodic function is the horizontal line midway
  between the function's minimum and maximum values.

  If $y = f(t)$ is periodic and $f_{max}$ and $f_{min}$ are the maximum and
  minimum values of $f$, then the equation of the midline is:
  \[ y = \frac{f_{max} - f_min}{2} \].
\end{definition}

\begin{definition}[Amplitude]
  \label{def:amplitude}

  The \textbf{amplitude} of a period function is the distance between the
  function's maximum value and the midline (or the function's minimum value and
  the midline).
\end{definition}

\begin{exc}[Solution \ref{sol:midline_and_amplitude}]
  \label{exc:midline_and_amplitude}

  Find the midline and amplitude of the function graphed below:

  \begin{figure}[H]
    \centering

    \begin{tikzpicture}
      \begin{axis}[
          my axis style,
          width=\textwidth,
          height=.5\textwidth,
        ]
        \addplot[
          domain=-10:10,
          red,
          thick,
          <->
        ]
        {sin(deg(x*3.14))};
      \end{axis}
    \end{tikzpicture}

    \caption{}
    \label{fig:find_the_midline_and_period}
  \end{figure}
\end{exc}

\newpage
