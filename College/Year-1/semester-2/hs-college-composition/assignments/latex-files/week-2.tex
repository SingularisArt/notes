\begin{essay}
  When I was younger, about $8$ - $9$ years old, I enjoyed watching shows like
  \textbf{Popular Mechanics}, \textbf{National Geographic}, \textbf{etc}. But I
  regretted watching one of the \textbf{National Geographic} videos for almost $2$
  years. The video talked about sinkholes, which terrorized me. Since I was a very
  optimistic, naive little kid who didn't think anything was wrong in the world,
  this took its toll on me. My bubble of a world shook. The world wasn't all
  flowers and roses anymore. I created many dark ideas about how the world may
  end., from fire and flames to floods and light beaming from the sky. From that
  day on, I imagined every day as the last day. Every day, I would act as if it
  was my last day, spending countless hours reading about sinkholes. Every time
  I'd close my eyes, I would hear the piercing sound coming from the sinkholes and
  visually see it as if I was falling into one. My family forbade me from reading
  any more books on the matter. Eventually, like all little kids, I grew out of
  this phobia.
  
  The first example of an apocalyptic movie is \textbf{Train to Busan}. The main
  plot of the movie is about a father and his daughter trying to make their way
  across \textbf{South Korea}, via a train, which out of nowhere, a virus starts
  to spread among the passengers. The said virus kills the victim and makes them
  become a zombie. This is a special movie since it focuses on different aspects,
  which other zombie movies may not. It criticizes society on many levels, one
  being, how humans are and how an event so colossal as an apocalypse can show
  their true skin. Another key element that makes this movie so great is the
  likability of the characters. Usually, we see one main character who survives by
  sheer luck or by sacrificing everybody to save himself. In this movie, the
  clever ones are the people standing in the end if they decide not to sacrifice
  themselves to save everybody else.
  
  At first, I assumed that the term \textbf{Apocalypse} meant the end of the
  world. Although that definition is a worldwide accepted meaning, I believe there
  are other definitions or interpretations. One may be \textit{Apocalypse brings
  out the true nature of the human being, which helps you distinguish your friends
  and family from your enemies, it may bring on a new beginning.} The first
  section states that an apocalyptic event brings out the true nature of the human
  being, whether that is helping others and sacrificing yourself if needed or
  sacrificing others to help yourself. The second section states that there may be
  another chapter in the world's life, but since we humans are measly and
  pathetic, we may never comprehend that fact.
\end{essay}

\newpage

\section*{Letter of Reflection}
\label{sec:letter_of_reflection}

\begin{enumerate}
  \label{enum:letter_of_reflection}

  \item I'm going to add another paragraph, I just didn't have enough time.
  \item I would like to draw connections because right now, all of my stuff are
    all over the place.
  \item I'm going to re-structure the essay because I think right now, it's
    really messy
  \item I'm going to spend time re-phrasing things and removing all the bits and
    pieces that slow down my essay
\end{enumerate}

% section letter_of_reflection (end)
