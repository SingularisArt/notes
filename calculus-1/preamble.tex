%%%%%%%%%%%%%%%%%%%%%%%%%%%%%%%%%%%%%%%%%%%%%%%%%%%%%%%%%%%%%%%%%%%%%%%%%%%%%%%%
%                                                                              %
%                              Required Packages                               %
%                                                                              %
%%%%%%%%%%%%%%%%%%%%%%%%%%%%%%%%%%%%%%%%%%%%%%%%%%%%%%%%%%%%%%%%%%%%%%%%%%%%%%%%

% Set encoding style.
\usepackage[utf8]{inputenc}
% Gives us multiple colors.
\usepackage[usenames,dvipsnames,pdftex]{xcolor}
% Lets us style link colors.
\usepackage{hyperref}
% Lets us import images and graphics.
\usepackage{graphicx}
% Let's us modify list stuff.
\usepackage{enumitem}
% Lets us use figures in floating environments.
\usepackage{float}
% Lets us create multiple columns.
\usepackage{multicol}
\usepackage{multirow}
% Gives us better math syntax.
\usepackage{amsmath,amsfonts,mathtools,amsthm,amssymb}
% Lets us strike through text.
\usepackage{cancel}
% Lets us import pdf directly in our tex code.
\usepackage{pdfpages}
% Derivative stuff.
\usepackage{derivative}
% Lets us add vectors graphically.
\usepackage{physics}
% Table stuff.
\usepackage{tablists}
\usepackage{tabularx}
\usepackage{wasysym}


%%%%%%%%%%%%%%%%%%%%%%%%%%%%%%%%%%%%%%%%%%%%%%%%%%%%%%%%%%%%%%%%%%%%%%%%%%%%%%%%
%                                                                              %
%                                Basic Settings                                %
%                                                                              %
%%%%%%%%%%%%%%%%%%%%%%%%%%%%%%%%%%%%%%%%%%%%%%%%%%%%%%%%%%%%%%%%%%%%%%%%%%%%%%%%

%%%%%%%%%
% Tasks %
%%%%%%%%%

\usepackage{tasks}
\usepackage{extramarks}

\settasks{label=\bfseries\arabic*.),label-width=2em}

%%%%%%%%%%%
%  Table  %
%%%%%%%%%%%

\newcolumntype{C}{>{\Centering\arraybackslash}X}

\usepackage[font=bf]{caption}
\renewcommand\thetable{\Roman{table}}
% \renewcommand\qedsymbol{$\Laughey$}

\let\svlim\lim\def\lim{\svlim\limits}
\let\svsum\sum\def\sum{\svsum\limits}

%%%%%%%%%%%%%
%  Symbols  %
%%%%%%%%%%%%%

\let\implies\Rightarrow
\let\impliedby\Leftarrow
\let\iff\Leftrightarrow
\let\epsilon\varepsilon

%%%%%%%%%%%
%  Lists  %
%%%%%%%%%%%

\setlist[itemize,1]{label=--}
\setlist[itemize,2]{label=\textbullet}
\setlist[enumerate,1]{label=\protect\circled{\arabic*}}

%%%%%%%%%%%%
%  Tables  %
%%%%%%%%%%%%

\setlength{\tabcolsep}{5pt}
\renewcommand\arraystretch{1.5}

%%%%%%%%%%%%%%
%  SI Unitx  %
%%%%%%%%%%%%%%

\usepackage{siunitx}
\sisetup{
  locale = US,
  per-mode = symbol,
  propagate-math-font = true,
  reset-math-version = false,
  exponent-mode = engineering,
  round-mode = figures,
  round-precision = 3,
  drop-zero-decimal,
}

% Distance
\DeclareSIUnit{\millimeter}{mm}
\DeclareSIUnit{\centimeter}{cm}
\DeclareSIUnit{\decimeter}{dm}
\DeclareSIUnit{\inch}{in}
\DeclareSIUnit{\foot}{ft}
\DeclareSIUnit{\yard}{yd}
\DeclareSIUnit{\meter}{m}
\DeclareSIUnit{\kilometer}{km}
\DeclareSIUnit{\mile}{mi}

% Time
\DeclareSIUnit{\millisecond}{ms}
\DeclareSIUnit{\second}{sec}
\DeclareSIUnit{\minute}{min}
\DeclareSIUnit{\hour}{hr}
\DeclareSIUnit{\day}{d}
\DeclareSIUnit{\week}{wk}
\DeclareSIUnit{\month}{mos}
\DeclareSIUnit{\year}{yr}

% Weight
\DeclareSIUnit{\milligram}{mg}
\DeclareSIUnit{\gram}{g}
\DeclareSIUnit{\ounce}{oz}
\DeclareSIUnit{\pound}{lb}
\DeclareSIUnit{\kilogram}{kg}
\DeclareSIUnit{\ton}{t}

% Liquid
\DeclareSIUnit{\gallon}{gal}
\DeclareSIUnit{\liter}{L}
\DeclareSIUnit{\milliliter}{mL}

%%%%%%%%%%
%  TikZ  %
%%%%%%%%%%

\usepackage[framemethod=TikZ]{mdframed}
\usepackage{tikz}
\usepackage{animate}
\usepackage{tikz-cd}
\usepackage{bm}

\usetikzlibrary{
  intersections,
  angles,
  quotes,
  calc,
  positioning,
  3d,
  arrows,
  arrows.meta,
  patterns,
}

\tikzset{>=stealth}

\tikzstyle{vector label} = [midway,fill=white,sloped]
\tikzstyle{vector}=[->,very thick]
\tikzstyle{construction} = [->,thin,dashed,draw=green!45!black]

%%%%%%%%%%%%%%%
%  PGF Plots  %
%%%%%%%%%%%%%%%

\usepackage{pgfplots}
\pgfplotsset{compat=1.18}

\usepgfplotslibrary{patchplots}

\pgfplotsset{pccplot/.style={color=red,mark=none,line width=1pt,<->,solid}}
\pgfplotsset{asymptote/.style={color=gray,mark=none,line width=1pt,<->,dashed}}
\pgfplotsset{soldot/.style={color=red,only marks,mark=*}}
\pgfplotsset{holdot/.style={color=red,fill=white,only marks,mark=*}}
\pgfplotsset{blankgraph/.style={xmin=-10,xmax=10,ymin=-10,ymax=10,axis line style= {-, draw opacity=0 },axis lines=box,major tick length=0mm,xtick={-10,-9,...,10},ytick={-10,-9,...,10},grid=major,yticklabels={,,},xticklabels={,,},minor xtick=,minor ytick=,xlabel={},ylabel={},width=0.75\textwidth,grid style={solid,gray!40}}}

\pgfplotscreateplotcyclelist{pccstylelist}{
  pccplot \\
  color=blue,mark=none,line width=1pt,<->,dashdotted \\
  color=gray,mark=none,line width=1pt,<->,dashdotdotted \\
}

\def\axisdefaultwidth{175pt}
\def\axisdefaultheight{\axisdefaultwidth}

\pgfplotsset{every axis/.append style={
    axis x line=middle,
    axis y line=middle,
    axis line style={<->},
    xlabel={$x$},
    ylabel={$y$},
    xmin = -7,xmax = 7,
    ymin = -7,ymax = 7,
    yticklabel style={inner sep=0.333ex},
    minor xtick = {-7,-6,...,7},
    minor ytick = {-7,-6,...,7},
    scale only axis,
    cycle list name=pccstylelist,
    tick label style={font=\footnotesize},
    legend cell align=left,
    grid = minor,
    grid style = {solid,gray!40},
    try min ticks=6,
  },
  framed/.style={axis background/.style = {draw=gray}}
}

% framing the graphs
\pgfplotsset{axis background/.style={draw=gray}}

%%%%%%%%%%%%%%%%%%%%%%%
%  Center Title Page  %
%%%%%%%%%%%%%%%%%%%%%%%

\usepackage{titling}
\renewcommand\maketitlehooka{\null\mbox{}\vfill}
\renewcommand\maketitlehookd{\vfill\null}

%%%%%%%%%%%%%%%%%%%%%%%%%%%%%%%%%%%%%%%%%%%%%%%%%%%%%%%
%  Create a grey background in the middle of the PDF  %
%%%%%%%%%%%%%%%%%%%%%%%%%%%%%%%%%%%%%%%%%%%%%%%%%%%%%%%

\usepackage{eso-pic}

%%%%%%%%%%%%%%%%%%%%%%%%%%%%%%%%%%%%%%%%%%%%%%%%%%%%%%%%%%%%%%%%%%%%%%%%%%%%%%%%
% EXAMPLE:                                                                     %
% ---------------------------------------------------------------------------- %
%                                                                              %
% PARAMETERS:                                                                  %
% ---------------------------------------------------------------------------- %
%                                                                              %
% DESCRIPTION:                                                                 %
% ---------------------------------------------------------------------------- %
%%%%%%%%%%%%%%%%%%%%%%%%%%%%%%%%%%%%%%%%%%%%%%%%%%%%%%%%%%%%%%%%%%%%%%%%%%%%%%%%
\newcommand\definegraybackground{
  \definecolor{reallylightgray}{HTML}{FAFAFA}
  \AddToShipoutPicture{
    \ifthenelse{\isodd{\thepage}}{
      \AtPageLowerLeft{
        \put(\LenToUnit{\dimexpr\paperwidth-222pt},0){
          \color{reallylightgray}\rule{222pt}{297mm}
        }
      }
    }
    {
      \AtPageLowerLeft{
        \color{reallylightgray}\rule{222pt}{297mm}
      }
    }
  }
}

%%%%%%%%%%%%%%%%%%%
%  Footnote Line  %
%%%%%%%%%%%%%%%%%%%

\renewcommand\footnoterule{\hrule\vspace{0.1cm}}

%%%%%%%%%%%%%%%%%%%%%%%%
%  Modify Links Color  %
%%%%%%%%%%%%%%%%%%%%%%%%

\hypersetup{
  % Enable highlighting links.
  colorlinks,
  % Change the color of links to blue.
  linkcolor=blue,
  % Change the color of citations to black.
  citecolor={black},
  % Change the color of url's to blue with some black.
  urlcolor={blue!80!black}
}

%%%%%%%%%%%%%%%%%%
% Fix WrapFigure %
%%%%%%%%%%%%%%%%%%

\newcommand{\wrapfill}{\par\ifnum\value{WF@wrappedlines}>0
    \parskip=0pt
    \addtocounter{WF@wrappedlines}{-1}%
    \null\vspace{\arabic{WF@wrappedlines}\baselineskip}%
    \WFclear
\fi}

%%%%%%%%%%%%%%%%%
% Multi Columns %
%%%%%%%%%%%%%%%%%

\let\multicolmulticols\multicols
\let\endmulticolmulticols\endmulticols

\RenewDocumentEnvironment{multicols}{mO{}}
{%
  \ifnum#1=1
    #2%
  \else
    \multicolmulticols{#1}[#2]
  \fi
}
{%
  \ifnum#1=1
\else
  \endmulticolmulticols
\fi
}

\newlength{\thickarrayrulewidth}
\setlength{\thickarrayrulewidth}{5\arrayrulewidth}

%%%%%%%%%%%%%%%%%%%
%  Caption Setup  %
%%%%%%%%%%%%%%%%%%%

\captionsetup[figure]{font=small}
\captionsetup{justification=centering}


%%%%%%%%%%%%%%%%%%%%%%%%%%%%%%%%%%%%%%%%%%%%%%%%%%%%%%%%%%%%%%%%%%%%%%%%%%%%%%%%
%                                                                              %
%                           School Specific Commands                           %
%                                                                              %
%%%%%%%%%%%%%%%%%%%%%%%%%%%%%%%%%%%%%%%%%%%%%%%%%%%%%%%%%%%%%%%%%%%%%%%%%%%%%%%%

%%%%%%%%%%%%%%%%%%%%%%
%  Helpful Commands  %
%%%%%%%%%%%%%%%%%%%%%%

\makeatletter

\newcommand\resetcounters{
  \setcounter{subsection}{0}
  \setcounter{subsubsection}{0}
  \setcounter{paragraph}{0}
  \setcounter{subparagraph}{0}
  \setcounter{theorem}{0}
  \setcounter{claim}{0}
  \setcounter{corollary}{0}
  \setcounter{lemma}{0}
  \setcounter{exercise}{0}
  \setcounter{definition}{0}
}

%%%%%%%%%%%%%%%%%%%%%%%%%%%%%%%%%%%%%%%%%%%%%%%%%%%%%%%%%%%%%%%%%%%%%%%%%%%%%%%%
% EXAMPLE:                                                                     %
% ---------------------------------------------------------------------------- %
% 1. \lec{1}                                                                   %
% 2. \lec{4}                                                                   %
%                                                                              %
% PARAMETERS:                                                                  %
% ---------------------------------------------------------------------------- %
% 1. Lecture number.                                                           %
%                                                                              %
% DESCRIPTION:                                                                 %
% ---------------------------------------------------------------------------- %
% It loads the lecture provided. It looks into the `lectures/` folder. If you  %
% provided a number less than 10, it'll use 0 and you're number. If you        %
% provided, 4 for example, it'll try to include the file `lectures/lec-04.tex`.%
% It also sets the section number to the provided number and resets all        %
% counters back to zero.                                                       %
%%%%%%%%%%%%%%%%%%%%%%%%%%%%%%%%%%%%%%%%%%%%%%%%%%%%%%%%%%%%%%%%%%%%%%%%%%%%%%%%
\def\@lecnum{}
\newcommand\lec[1]{%
  \ifnum #1<10
    % If the lecture number passed is less than 10, add a 0 in front of the
    % digit.
    \def\@lecnum{0#1}
  \else
    \def\@lecnum{#1}
  \fi
  % Set the section counter to the number passed.
  \setcounter{section}{#1}
  % Reset all counters.
  \resetcounters
  % Include the lecture.
  \input{lectures/lec-\@lecnum.tex}
}

%%%%%%%%%%%%%%%%%%%%%
%  Lecture Command  %
%%%%%%%%%%%%%%%%%%%%%

\usepackage{ifthen}
\usepackage{xifthen}

%%%%%%%%%%%%%%%%%%%%%%%%%%%%%%%%%%%%%%%%%%%%%%%%%%%%%%%%%%%%%%%%%%%%%%%%%%%%%%%%
% EXAMPLE:                                                                     %
% ---------------------------------------------------------------------------- %
% 1. \lecture{Oct 17 2022 Mon (08:46:48)}{Lecture Title}                       %
% 2. \lecture{Oct 17 2022 Mon (08:46:48)}{}                                    %
%                                                                              %
% PARAMETERS:                                                                  %
% ---------------------------------------------------------------------------- %
% 1. Time and date of lecture.                                                 %
% 2. Lecture Title.                                                            %
%                                                                              %
% DESCRIPTION:                                                                 %
% ---------------------------------------------------------------------------- %
% Display the information like the following:                                  %
%                                                                              %
%                                                   Oct 17 2022 Mon (08:49:10) %
% ---------------------------------------------------------------------------- %
% Lecture 1: Lecture Title                                                     %
%                                                                              %
% You don't need to provide a lecture number, since it uses the lecture number %
% from the file as the lecture number.                                         %
%%%%%%%%%%%%%%%%%%%%%%%%%%%%%%%%%%%%%%%%%%%%%%%%%%%%%%%%%%%%%%%%%%%%%%%%%%%%%%%%
\def\@lecture{}
\newcommand\lecture[2]{
  \ifthenelse{\isempty{#2}}{
    \def\@lecture{Lecture \arabic{section}}
  }{
    \def\@lecture{Lecture \arabic{section}: #2}
  }

  \if@twocolumn
    \twocolumn[
    \hfill\footnotesize{#1}
    \hrule
    \vspace*{-0.3cm}
    \section*{\@lecture}
    \vspace{0.2cm}
    ]
  \else
    \hfill\footnotesize{#1}
    \hrule
    \vspace*{-0.3cm}
    \section*{\@lecture}
  \fi

  \addcontentsline{toc}{section}{\@lecture}
}

%%%%%%%%%%%%%%%%%
% Fancy Headers %
%%%%%%%%%%%%%%%%%

\usepackage{fancyhdr}

%%%%%%%%%%%%%%%%%%%%%%%%%%%%%%%%%%%%%%%%%%%%%%%%%%%%%%%%%%%%%%%%%%%%%%%%%%%%%%%%
% EXAMPLE:                                                                     %
% ---------------------------------------------------------------------------- %
% 1. \forcenewpage                                                             %
%                                                                              %
% DESCRIPTION:                                                                 %
% ---------------------------------------------------------------------------- %
% Force a new page, because sometimes latex doesn't do it.                     %
%%%%%%%%%%%%%%%%%%%%%%%%%%%%%%%%%%%%%%%%%%%%%%%%%%%%%%%%%%%%%%%%%%%%%%%%%%%%%%%%
\newcommand\forcenewpage{\clearpage\mbox{~}\clearpage\newpage}

%%%%%%%%%%%%%%%%%%%%%%%%%%%%%%%%%%%%%%%%%%%%%%%%%%%%%%%%%%%%%%%%%%%%%%%%%%%%%%%%
% EXAMPLE:                                                                     %
% ---------------------------------------------------------------------------- %
% 1. \createintro                                                              %
%                                                                              %
% DESCRIPTION:                                                                 %
% ---------------------------------------------------------------------------- %
% This command does a million things.                                          %
%                                                                              %
% First, it creates a title page with the following properties:                %
% ---------------------------------------------------------------------------- %
% 1. It sets the numbering style to roman, removes the header line, and creates%
%    a nice footer center style.                                               %
% 2. It checks if you passed the `twocolumn` parameter. If so, it sets the     %
%    title page to one column.                                                 %
% 3. It checks if a intro.tex file exists. If so, it includes it.              %
% 4. It creates a 3d box with more information.                                %
% 5. It includes the table of contents.                                        %
% 6. It checks if you passed the `twocolumn` parameter. If so, it sets         %
%    everything back to two column. Otherwise, it just creates a new page.     %
%                                                                              %
% Now, it gets everything prepared to include the actual notes.                %
% ---------------------------------------------------------------------------- %
% 1. It sets the numbering style back to arabic, adds the header line, and     %
%    changes the fancy style.                                                  %
%       1. It adds the current lecture title in the top right.                 %
%       2. It adds the author's name in the top left.                          %
%       3. It adds the page number in the bottom center.                       %
% 2. It checks if you passed the `graygf` parameter. If so, it creates a nice  %
%    transparent gray color on the right of all left pages and on the left of  %
%    all right pages.                                                          %
%%%%%%%%%%%%%%%%%%%%%%%%%%%%%%%%%%%%%%%%%%%%%%%%%%%%%%%%%%%%%%%%%%%%%%%%%%%%%%%%
\newcommand\createintro{
  \@ifclasswith\class{twocolumn}{\onecolumn}{}
  \pagenumbering{roman}

  % Create title page.
  \begin{center}
    {\LARGE\@title} \\
    {\Large\vspace{0.25cm}\@author} \\
    {\large\vspace{0.25cm}\@date}
  \end{center}

  % Check if the intro.tex file exists.
  % If it does, include it, otherwise, just ignore.
  \IfFileExists{./intro.tex}{\input{intro.tex}}{}

  % Set the pagestyle to fancy.
  \pagestyle{fancy}
  % Remove the header line.
  \renewcommand\headrulewidth{0pt}

  % Reset fancyhead styles.
  \fancyhead{}
  % Add a fancyfoot center style.
  \fancyfoot[C]{
    \textit{For more notes like this, visit \href{\linktootherpages}{\shortlinkname}}.
  }

  % Create a box with more information.
  \vspace{0.5cm}
  \begin{tcolorbox}[enhanced,colback=white,center upper,size=fbox, drop shadow southwest,sharp corners]
    \term: \academicyear, \\
    Last Update: \today, \\
    \faculty, \location.
  \end{tcolorbox}

  % Create a table of contents.
  \tableofcontents

  \@ifclasswith\class{twocolumn}{%
    % If user passed twocolumn as a parameter to \documentclass, then set the
    % rest of the page to two column.
    \twocolumn%
  }{%
    % Otherwise, just create a new page.
    \newpage%
  }

  % Change the numbering style back to arabic from roman.
  \pagenumbering{arabic}
  % Reset page numbers back to 1.
  \setcounter{page}{1}

  % Add back the header line.
  \renewcommand\headrulewidth{0.4pt}
  % Add the lecture name in the top right.
  \fancyhead[R]{\@lecture}
  % Add the author name in the top left.
  \fancyhead[L]{\@author}
  % Add the page number in the bottom center.
  \fancyfoot[C]{\thepage}
  \@ifclasswith\class{grayfg}{%
    % If user passed twocolumn as a parameter to \documentclass, then set the
    % rest of the page to two column.
    \definegraybackground%
  }{}
}

%%%%%%%%%%%%%%%%%%%%
%  Import Figures  %
%%%%%%%%%%%%%%%%%%%%

\usepackage{import}
\pdfminorversion=7

%%%%%%%%%%%%%%%%%%%%%%%%%%%%%%%%%%%%%%%%%%%%%%%%%%%%%%%%%%%%%%%%%%%%%%%%%%%%%%%%
% EXAMPLE:                                                                     %
% ---------------------------------------------------------------------------- %
% 1. \incfig{airplane}                                                         %
% 2. \incfig[0.5]{airplane}                                                    %
%                                                                              %
% PARAMETERS:                                                                  %
% ---------------------------------------------------------------------------- %
% 1. (Optional) The width of the image with respect to textwidth. If you put   %
%    0.3, it'll be 0.3 * \textwidth.                                           %
% 2. The figure name.                                                          %
%                                                                              %
% DESCRIPTION:                                                                 %
% ---------------------------------------------------------------------------- %
% It looks for a file in:                                                      %
%   ./figures/lec-(current lecture number)/fig-name.pdf_tex                    %
% If you're current lecture was 4, and the figure name was airplane, here's how%
% it would look like:                                                          %
%   ./figures/lec-04/airplane.pdf_tex                                          %
%%%%%%%%%%%%%%%%%%%%%%%%%%%%%%%%%%%%%%%%%%%%%%%%%%%%%%%%%%%%%%%%%%%%%%%%%%%%%%%%
\newcommand\incfig[2][1]{
  \def\figlocation{./figures/lec-\@lecnum}
  \def\svgwidth{#1\columnwidth}
  \import{\figlocation}{#2.pdf_tex}
}

\makeatother

%%%%%%%%%%%%%%%%%%%%%%%%%%%%%%%%%%%%%%%%%%%%%%%%%%%%%%%%%%%%%%%%%
%  Add vectors using the parallelogram and head to tail method  %
%%%%%%%%%%%%%%%%%%%%%%%%%%%%%%%%%%%%%%%%%%%%%%%%%%%%%%%%%%%%%%%%%

\usepackage{pdftexcmds}

%%%%%%%%%%%%%%%%%%%%%%%%%%%%%%%%%%%%%%%%%%%%%%%%%%%%%%%%%%%%%%%%%%%%%%%%%%%%%%%%
% EXAMPLE:                                                                     %
% ---------------------------------------------------------------------------- %
%                                                                              %
% PARAMETERS:                                                                  %
% ---------------------------------------------------------------------------- %
%                                                                              %
% DESCRIPTION:                                                                 %
% ---------------------------------------------------------------------------- %
%%%%%%%%%%%%%%%%%%%%%%%%%%%%%%%%%%%%%%%%%%%%%%%%%%%%%%%%%%%%%%%%%%%%%%%%%%%%%%%%
\newcommand\parallelogramRule[6]{
  \begin{tikzpicture}[inner sep=1pt,>=stealth]
    \draw[<->] (#1)--(#2) node[right]{$x$};
    \draw[<->] (#3)--(#4) node[above]{$y$};

    \coordinate (O) at (#1);
    \coordinate (A) at (#5);
    \coordinate (B) at (#6);
    \coordinate (A+B) at ($(A)+(B)$);

    \draw[vector,blue] (O)--(A) node[vector label] {$\vec{x}$};
    \draw[vector,red] (O)--(B) node[vector label] {$\vec{y}$};

    \draw[construction] (A)--(A+B);
    \draw[construction] (B)--(A+B);

    \draw[vector,purple] (O)--(A+B) node[vector label] {$\vec{x}+\vec{y}$};
  \end{tikzpicture}
}

%%%%%%%%%%%%%%%%%%%%%%%%%%%%%%%%%%%%%%%%%%%%%%%%%%%%%%%%%%%%%%%%%%%%%%%%%%%%%%%%
% EXAMPLE:                                                                     %
% ---------------------------------------------------------------------------- %
%                                                                              %
% PARAMETERS:                                                                  %
% ---------------------------------------------------------------------------- %
%                                                                              %
% DESCRIPTION:                                                                 %
% ---------------------------------------------------------------------------- %
%%%%%%%%%%%%%%%%%%%%%%%%%%%%%%%%%%%%%%%%%%%%%%%%%%%%%%%%%%%%%%%%%%%%%%%%%%%%%%%%
\newcommand\headToTailRule[6]{
  \begin{tikzpicture}[inner sep=1pt,>=stealth]
    \draw[<->] (#1)--(#2) node[right]{$x$};
    \draw[<->] (#3)--(#4) node[above]{$y$};

    \coordinate (O) at (#1);
    \coordinate (A) at (#5);
    \coordinate (B) at (#6);
    \coordinate (A+B) at ($(A)+(B)$);

    \draw[vector,blue] (O)--(A) node[vector label] {$\vec{x}$};
    \draw[construction] (O)--(B);

    \draw[vector,red] (A)--(A+B) node[vector label] {$\vec{y}$};
    \draw[construction] (B)--(A+B);

    \draw[vector,purple] (O)--(A+B) node[vector label] {$\vec{x}+\vec{y}$};
  \end{tikzpicture}
}

\makeatletter

%%%%%%%%%%%%%%%%%%%%%%%%%%%%%%%%%%%%%%%%%%%%%%%%%%%%%%%%%%%%%%%%%%%%%%%%%%%%%%%%
% EXAMPLE:                                                                     %
% ---------------------------------------------------------------------------- %
%                                                                              %
% PARAMETERS:                                                                  %
% ---------------------------------------------------------------------------- %
%                                                                              %
% DESCRIPTION:                                                                 %
% ---------------------------------------------------------------------------- %
%%%%%%%%%%%%%%%%%%%%%%%%%%%%%%%%%%%%%%%%%%%%%%%%%%%%%%%%%%%%%%%%%%%%%%%%%%%%%%%%
\newcommand\addVectors[7][parallelogram]{
  \ifnum\pdf@strcmp{\unexpanded{#1}}{parallelogram}=0%
     \expandafter\@firstoftwo
  \fi
  {\parallelogramRule{#2}{#3}{#4}{#5}{#6}{#7}}
  \ifnum\pdf@strcmp{\unexpanded{#1}}{headToTail}=0%
     \expandafter\@firstoftwo
  \fi
  {\headToTailRule{#2}{#3}{#4}{#5}{#6}{#7}}
}

\makeatother

%%%%%%%%%%%%
%  Circle  %
%%%%%%%%%%%%

%%%%%%%%%%%%%%%%%%%%%%%%%%%%%%%%%%%%%%%%%%%%%%%%%%%%%%%%%%%%%%%%%%%%%%%%%%%%%%%%
% EXAMPLE:                                                                     %
% ---------------------------------------------------------------------------- %
%                                                                              %
% PARAMETERS:                                                                  %
% ---------------------------------------------------------------------------- %
%                                                                              %
% DESCRIPTION:                                                                 %
% ---------------------------------------------------------------------------- %
%%%%%%%%%%%%%%%%%%%%%%%%%%%%%%%%%%%%%%%%%%%%%%%%%%%%%%%%%%%%%%%%%%%%%%%%%%%%%%%%
\newcommand*\circled[1]{\tikz[baseline=(char.base)]{
  \node[shape=circle,draw,inner sep=1pt] (char) {#1};}
}

%%%%%%%%%%%%%
%  Correct  %
%%%%%%%%%%%%%

\definecolor{correct}{HTML}{009900}

%%%%%%%%%%%%%%%%%%%%%%%%%%%%%%%%%%%%%%%%%%%%%%%%%%%%%%%%%%%%%%%%%%%%%%%%%%%%%%%%
% EXAMPLE:                                                                     %
% ---------------------------------------------------------------------------- %
%                                                                              %
% PARAMETERS:                                                                  %
% ---------------------------------------------------------------------------- %
%                                                                              %
% DESCRIPTION:                                                                 %
% ---------------------------------------------------------------------------- %
%%%%%%%%%%%%%%%%%%%%%%%%%%%%%%%%%%%%%%%%%%%%%%%%%%%%%%%%%%%%%%%%%%%%%%%%%%%%%%%%
\newcommand\correct[2]{{\color{red}{#1 }}\ensuremath{\to}{\color{correct}{ #2}}}

%%%%%%%%%%%%%%%
%  Important  %
%%%%%%%%%%%%%%%

%%%%%%%%%%%%%%%%%%%%%%%%%%%%%%%%%%%%%%%%%%%%%%%%%%%%%%%%%%%%%%%%%%%%%%%%%%%%%%%%
% EXAMPLE:                                                                     %
% ---------------------------------------------------------------------------- %
%                                                                              %
% PARAMETERS:                                                                  %
% ---------------------------------------------------------------------------- %
%                                                                              %
% DESCRIPTION:                                                                 %
% ---------------------------------------------------------------------------- %
%%%%%%%%%%%%%%%%%%%%%%%%%%%%%%%%%%%%%%%%%%%%%%%%%%%%%%%%%%%%%%%%%%%%%%%%%%%%%%%%
\newcommand\imp[1]{{\color{red}#1}}

%%%%%%%%%%%%%%%%%%%
%  Todo Commands  %
%%%%%%%%%%%%%%%%%%%

\usepackage[colorinlistoftodos]{todonotes}

\makeatletter

\@ifclasswith\class{working}{
  \newcommand\unsure[2][]{\todo[linecolor=red,backgroundcolor=red!25,bordercolor=red,#1]{#2}}
  \newcommand\change[2][]{\todo[linecolor=yellow,backgroundcolor=yellow!25,bordercolor=yellow,#1]{#2}}
  \newcommand\add[2][]{\todo[linecolor=blue,backgroundcolor=blue!25,bordercolor=blue,#1]{#2}}
  \newcommand\info[2][]{\todo[linecolor=OliveGreen,backgroundcolor=OliveGreen!25,bordercolor=OliveGreen,#1]{#2}}
  \newcommand\improvement[2][]{\todo[linecolor=Plum,backgroundcolor=Plum!25,bordercolor=Plum,#1]{#2}}

  \newcommand\listnotes{
    \newpage
    \listoftodos[Notes]
  }
}{
  \newcommand\unsure[2][]{}
  \newcommand\change[2][]{}
  \newcommand\info[2][]{}
  \newcommand\improvement[2][]{}

  \newcommand\listnotes{}
}

\makeatother


%%%%%%%%%%%%%%%%%%%%%%%%%%%%%%%%%%%%%%%%%%%%%%%%%%%%%%%%%%%%%%%%%%%%%%%%%%%%%%%%
%                                                                              %
%                                 Environments                                 %
%                                                                              %
%%%%%%%%%%%%%%%%%%%%%%%%%%%%%%%%%%%%%%%%%%%%%%%%%%%%%%%%%%%%%%%%%%%%%%%%%%%%%%%%

\usepackage{varwidth}
\usepackage{thmtools}
\usepackage[most,many,breakable]{tcolorbox}

\tcbuselibrary{theorems,skins,hooks}

%%%%%%%%%%%%%%%%%%%
%  Define Colors  %
%%%%%%%%%%%%%%%%%%%

\definecolor{myblue}{RGB}{45, 111, 177}
\definecolor{mygreen}{RGB}{56, 140, 70}
\definecolor{myred}{RGB}{199, 68, 64}
\definecolor{mypurple}{RGB}{197, 92, 212}

\definecolor{theorem}{HTML}{00007B}
\definecolor{example}{HTML}{2A7F7F}
\definecolor{prop}{HTML}{191971}
\definecolor{lemma}{HTML}{983b0f}
\definecolor{exercise}{HTML}{88D6D1}

\colorlet{definition}{mygreen!85!black}
\colorlet{claim}{mygreen!85!black}
\colorlet{corollary}{mypurple!85!black}
\colorlet{proof}{theorem}
\colorlet{notation}{theorem}
\colorlet{remark}{example}

%%%%%%%%%%%%%%%%%%%%%%%%%%%%%%%%%%%%%%%%%%%%%%%%%%%%%%%%%
%  Create Environments Styles Based on Given Parameter  %
%%%%%%%%%%%%%%%%%%%%%%%%%%%%%%%%%%%%%%%%%%%%%%%%%%%%%%%%%

\mdfsetup{skipabove=1em,skipbelow=0em}

%%%%%%%%%%%%%%%%%%%%%%
%  Helpful Commands  %
%%%%%%%%%%%%%%%%%%%%%%

%%%%%%%%%%%%%%%%%%%%%%%%%%%%%%%%%%%%%%%%%%%%%%%%%%%%%%%%%%%%%%%%%%%%%%%%%%%%%%%%
% EXAMPLE:                                                                     %
% ---------------------------------------------------------------------------- %
%                                                                              %
% PARAMETERS:                                                                  %
% ---------------------------------------------------------------------------- %
%                                                                              %
% DESCRIPTION:                                                                 %
% ---------------------------------------------------------------------------- %
%%%%%%%%%%%%%%%%%%%%%%%%%%%%%%%%%%%%%%%%%%%%%%%%%%%%%%%%%%%%%%%%%%%%%%%%%%%%%%%%
\newcommand\createnewtheoremstyle[3]{
  \declaretheoremstyle[
  headfont=\bfseries\sffamily, bodyfont=\normalfont, #2,
  mdframed={
    #3,
  },
  ]{#1}
}

%%%%%%%%%%%%%%%%%%%%%%%%%%%%%%%%%%%%%%%%%%%%%%%%%%%%%%%%%%%%%%%%%%%%%%%%%%%%%%%%
% EXAMPLE:                                                                     %
% ---------------------------------------------------------------------------- %
%                                                                              %
% PARAMETERS:                                                                  %
% ---------------------------------------------------------------------------- %
%                                                                              %
% DESCRIPTION:                                                                 %
% ---------------------------------------------------------------------------- %
%%%%%%%%%%%%%%%%%%%%%%%%%%%%%%%%%%%%%%%%%%%%%%%%%%%%%%%%%%%%%%%%%%%%%%%%%%%%%%%%
\newcommand\createnewcoloredtheoremstyle[4]{
  \declaretheoremstyle[
  headfont=\bfseries\sffamily\color{#2}, bodyfont=\normalfont, #3,
  mdframed={
    linewidth=2pt,
    rightline=false, leftline=true, topline=false, bottomline=false,
    linecolor=#2, backgroundcolor=#2!5, #4,
  },
  ]{#1}
}

%%%%%%%%%%%%%%%%%%%%%%%%%%%%%%%%%%%
%  Create the Environment Styles  %
%%%%%%%%%%%%%%%%%%%%%%%%%%%%%%%%%%%

\makeatletter
\@ifclasswith\class{nocolor}{
  \createnewtheoremstyle{thmclaimbox}{}{}
  \createnewtheoremstyle{thmcorollarybox}{}{}
  \createnewtheoremstyle{thmdefinitionbox}{}{}
  \createnewtheoremstyle{thmexercisebox}{}{}
  \createnewtheoremstyle{thmexamplebox}{}{}
  \createnewtheoremstyle{thmexplanationbox}{}{rightline=false, leftline=true, topline=false, bottomline=false}
  \createnewtheoremstyle{thmlemmabox}{}{}
  \createnewtheoremstyle{thmnotationbox}{}{rightline=false, leftline=true, topline=false, bottomline=false}
  \createnewtheoremstyle{thmpropbox}{}{}
  \createnewtheoremstyle{thmremarkbox}{}{rightline=false, leftline=true, topline=false, bottomline=false}
  \createnewtheoremstyle{thmreplacementproofbox}{qed=\qedsymbol}{rightline=false, leftline=true, topline=false, bottomline=false}
  \createnewtheoremstyle{thmtheorembox}{}{}
}{
  \createnewcoloredtheoremstyle{thmclaimbox}{claim}{}{}
  \createnewcoloredtheoremstyle{thmcorollarybox}{corollary}{}{}
  \createnewcoloredtheoremstyle{thmdefinitionbox}{definition}{}{}
  \createnewcoloredtheoremstyle{thmexamplebox}{example}{}{
    rightline=true, leftline=true, topline=true, bottomline=true
  }
  \createnewcoloredtheoremstyle{thmexercisebox}{exercise}{}{}
  \createnewcoloredtheoremstyle{thmexplanationbox}{example}{qed=\qedsymbol}{backgroundcolor=white}
  \createnewcoloredtheoremstyle{thmlemmabox}{lemma}{}{}
  \createnewcoloredtheoremstyle{thmnotationbox}{notation}{}{backgroundcolor=white}
  \createnewcoloredtheoremstyle{thmpropbox}{prop}{}{}
  \createnewcoloredtheoremstyle{thmremarkbox}{remark}{}{backgroundcolor=white}
  \createnewcoloredtheoremstyle{thmreplacementproofbox}{proof}{qed=\qedsymbol}{backgroundcolor=white}
  \createnewcoloredtheoremstyle{thmtheorembox}{theorem}{}{}
}
\makeatother

%%%%%%%%%%%%%%%%%%%%%%%%%%%%%
%  Create the Environments  %
%%%%%%%%%%%%%%%%%%%%%%%%%%%%%

\declaretheorem[numberwithin=section, style=thmclaimbox,            name=Claim]      {claim}
\declaretheorem[numberwithin=section, style=thmcorollarybox,        name=Corollary]  {corollary}
\declaretheorem[numberwithin=section, style=thmdefinitionbox,       name=Definition] {definition}
\declaretheorem[numbered=no,          style=thmexamplebox,          name=Example]    {example}
\declaretheorem[numberwithin=section, style=thmexercisebox,         name=Exercise]   {exercise}
\declaretheorem[numbered=no,          style=thmexplanationbox,      name=Proof]      {expl}
\declaretheorem[numberwithin=section, style=thmlemmabox,            name=Lemma]      {lemma}
\declaretheorem[numbered=no,          style=thmnotationbox,         name=Notation]   {notation}
\declaretheorem[numberwithin=section, style=thmpropbox,             name=Proposition]{prop}
\declaretheorem[numbered=no,          style=thmremarkbox,           name=Remark]     {remark}
\declaretheorem[numbered=no,          style=thmreplacementproofbox, name=Proof]      {replacementproof}
\declaretheorem[numberwithin=section, style=thmtheorembox,          name=Theorem]    {theorem}

%%%%%%%%%%%%%%%%%%%%%%%%%%%%%%%%%%%%%%%%%%%%%%%%%%%%%%%%%%%%%%%%%%%%%
%  Create Question and Solution Environments with or without color  %
%%%%%%%%%%%%%%%%%%%%%%%%%%%%%%%%%%%%%%%%%%%%%%%%%%%%%%%%%%%%%%%%%%%%%

\makeatletter
\@ifclasswith\class{nocolor}{
  \createnewtheoremstyle{thmquestionbox}{}{}
  \createnewtheoremstyle{thmsolutionbox}{}{}

  \declaretheorem[numberwithin=section, style=thmquestionbox,   name=Question]{question}
  \declaretheorem[numberwithin=section, style=thmsolutionbox,   name=Solution]{solution}

  \AtBeginEnvironment{solution}{
    \vspace{-10pt}
  }
}{
  \newtcbtheorem{Question}{Question}{enhanced,
    breakable,
    colback=white,
    colframe=myblue!80!black,
    attach boxed title to top left={yshift*=-\tcboxedtitleheight},
    fonttitle=\bfseries,
    title=\textbf{Question.},
    boxed title size=title,
    boxed title style={%
      sharp corners,
      rounded corners=northwest,
      colback=tcbcolframe,
      boxrule=0pt,
    },
    underlay boxed title={%
      \path[fill=tcbcolframe] (title.south west)--(title.south east)
      to[out=0, in=180] ([xshift=5mm]title.east)--
      (title.center-|frame.east)
      [rounded corners=\kvtcb@arc] |-
      (frame.north) -| cycle;
    },
    #1
  }{def}

  \NewDocumentEnvironment{question}{O{}O{}}
  {\begin{Question}{#1}{#2}}{\end{Question}}

  \newtcbtheorem{Solution}{Solution}{enhanced,
    breakable,
    colback=white,
    colframe=mygreen!80!black,
    attach boxed title to top left={yshift*=-\tcboxedtitleheight},
    fonttitle=\bfseries,
    title=\textbf{Solution.},
    boxed title size=title,
    boxed title style={%
      sharp corners,
      rounded corners=northwest,
      colback=tcbcolframe,
      boxrule=0pt,
    },
    underlay boxed title={%
      \path[fill=tcbcolframe] (title.south west)--(title.south east)
      to[out=0, in=180] ([xshift=5mm]title.east)--
      (title.center-|frame.east)
      [rounded corners=\kvtcb@arc] |-
      (frame.north) -| cycle;
    },
    #1
  }{def}

  \NewDocumentEnvironment{solution}{O{}O{}}
  {\vspace{-10pt}\begin{Solution}{#1}{#2}}{\end{Solution}}
}
\makeatother

%%%%%%%%%%%%%%%%%%%%%%%%%%%%
%  Edit Proof Environment  %
%%%%%%%%%%%%%%%%%%%%%%%%%%%%

\renewenvironment{proof}[1][\proofname]{\vspace{-10pt}\begin{replacementproof}}{\end{replacementproof}}
\newenvironment{explanation}[1][\proofname]{\vspace{-10pt}\begin{expl}}{\end{expl}}

%%%%%%%%%%%%%%%%%%%%%%%%%%%%%%%
%  Create Plain Environments  %
%%%%%%%%%%%%%%%%%%%%%%%%%%%%%%%

\theoremstyle{definition}

\newtheorem*{note}{Note}
\newtheorem*{pnotation}{Notation}
\newtheorem*{previouslyseen}{As previously seen}
\newtheorem*{problem}{Problem}
\newtheorem*{observe}{Observe}
\newtheorem*{property}{Property}
\newtheorem*{intuition}{Intuition}
\newtheorem*{questionexpl}{Explanation}

%%%%%%%%%%%%%%%%%%%%%%%%%
%  End of Environments  %
%%%%%%%%%%%%%%%%%%%%%%%%%

\AtEndEnvironment{exercise}{
  \addcontentsline{toc}{subsubsection}{Exercise \arabic{exercise}}
}
\AtEndEnvironment{question}{
  \addcontentsline{toc}{subsubsection}{Question \arabic{question}}
}
\AtEndEnvironment{solution}{
  \addcontentsline{toc}{subsubsection}{Solution \arabic{question}}
}
