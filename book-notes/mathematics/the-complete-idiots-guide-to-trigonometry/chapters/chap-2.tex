\chapter{Algebra Review}
\label{chap:algebra_tools_needed_to_study_trigonometry}

Basic algebra and trigonometry deal with two different areas of mathematics.
Many people believe that trigonometry is a completely different discipline than
algebra. But the truth is that isn't really the truth. You cannot fully
comprehend trigonometry without knowing the basis of algebra, making algebra a
prerequisite to trigonometry.

Algebra deals with finding the unknown variables and understanding functions --
whereas trigonometry explores aspects of triangles, which are their sides and
angles and finding a correlation between them. Algebra seeks to solve equations
composed of multiple terms and to find their roots, whereas trigonometry
focuses mainly on sin, cos, tan, degrees, radians, and polar coordinates.

It's helpful to imagine trigonometry as an extension to algebra because we
often treat trigonometric concepts in algebraic terms. For example, solving a
trigonometric equation is similar to solving an algebraic equation. You also
learn the basic rules of vital topics like polynomials, exponents, graphing and
much more in algebra. However, despite their similarities, there are still a
number of differences. They are still both prerequisites to more advanced
mathematical topics like calculus and differential equations.

\section{Cartesian Coordinates}
\label{sec:cartesian_coordinates}

In algebra, you use the standard Cartesian coordinate system when graphing
various functions. One of the most common methods of solving systems of
equations is graphing them, where the solution is the point of intersection in
all the lines.

\begin{definition}[Cartesian Coordinate System]
  \label{def:cartesian_coordinate_system}

  The \textbf{cartesian coordinate system} specifies each point on a plane by a
  pair of numerical coordinates, which are distances from the point to two
  fixed perpendicular lines, commonly referred to as the $x$-axis and $y$-axis.
\end{definition}

The simplest coordinate system is a number line on which you represent numbers.
Next, we represent ordered pairs of numbers in a plane on which two of these
lines are located.

In plane geometry, there are two axes at right angles that are usually called
$x$-axis and $y$-axis. The position of any point in the plane can be given by
its two coordinates $(x, y)$. These coordinates give the point's distance in
the $x$ and $y$ directions from the origin, which is the point of intersection
of the two axes. The origin is labeled with the pair of numbers both at zero --
$(0, 0)$.

\subsection{Writing Coordinates}
\label{sub_sec:writing_coordinates}

The following drawing illustrates the coordinate system with four labeled points.

\begin{note}
  Similar to the number line, you also can have negative values for $x$ and
  $y$.
\end{note}

\begin{figure}[ht]
  \centering
  \incfig[1]{coordinate_system}
  \caption{Coordinate System}
  \label{fig:coordinate_system}
\end{figure}

The origin $(0,0)$ is the starting point to determine distances when you need
to graph other points.  Two coordinates $(x,y)$ are also called an ordered pair
because the order is important. The value for the $x$-axis always comes first
and the value for the $y$-axis always comes second.

% subsection writing_coordinates (end)

\subsection{Four Quadrants}
\label{sub_sec:four_quadrants}

The $x$-axis and $y$-axis of the Cartesian coordinate system divide the plane
into four parts, which are called quadrants. They are numbered in a
counterclockwise direction using Roman numerals.

\begin{definition}[Quadrants]
  \label{def:quadrants}

  \textbf{Quadrants} are the four regions, which are divided by the coordinate
  plane.
\end{definition}

\begin{note}
  The values for the $x$-axis and $y$-axis are different for various
  coordinates in terms of their signs. For example, the values for the
  $x$-coordinates are positive in quadrants I and IV but negative in quadrants
  II and II. And the values for the $y$-coordinates are positive in quadrants
  II and II and negative in quadrants I and IV.
\end{note}

This information is useful when considering the signs of trigonometric
functions in different quadrants.

\begin{center}
  \begin{tabular}{|lccr|}
    \hline
      \textbf{Quadrant} & $x\qquad$ & $y\qquad$ & \textbf{Example} \\
    \hline
      I   & Positive & Positive & $(4,6)$ \\
      II  & Negative & Positive & $(-4,6)$ \\
      III & Negative & Negative & $(-4,-6)$ \\
      IV  & Positive & Negative & $(4,-6)$ \\
    \hline
  \end{tabular}
\end{center}

When you need to locate a spot in the real world, you have to use
three-dimensional coordinate system. In three dimensions, you have three axes
at right angles. $x$-axis, $y$-axis, and $z$-axis.

Coordinates can be continued into four and more dimensions, which help
mathematicians solve complex problems, because the higher you go, the easier
the problem becomes.

% subsection four_quadrants (end)

% section cartesian_coordinates (end)

\section{What Are Functions}
\label{sec:what_are_functions}

One of the most important tools in mathematics is a function. The term
\textit{function} first appeared in Gotta fried Leibniz's mathematics manuscript
in $1673$, which was later taken on by Leonhard Euler, who was one of the
greatest mathematicians ever. He broadened the concept of a function and made
it a core part of modern mathematics.

\begin{definition}[Function]
  \label{def:function}

  A \textbf{function} is a rule that assignments to every element in a set $D$ 
  exactly one element in a set $R$. The set $D$ is called the domain of the
  function, and the set $R$ is called the range.
\end{definition}

\begin{example}
  $A$ is a set that contains all the names of your families and friends and
  their birthdays. The pairing of names and birthdays form a relation. In this
  relation, as with functions, the pairs of names and birthdays are ordered.
  This would imply that the name will always be the first bit of information in
  the pair and birthday will always be the second bit of information in the
  pair. The set of all the starting data, which are the names is called the
  \textbf{domain} and the set of all the ending data, which are the birthdays
  is called the \textbf{range}. The domain is what you start with and the range
  is what you end up with.
\end{example}

\subsection{The Basics of Functions}
\label{sub_sec:the_basics_of_functions}

A function is a relation represented by graphs, tables, and equations.

\begin{definition}
  A \textbf{relation} is a relationship between two sets of information.

  Entities are \textbf{ordered} if the order that they are presented in
  matters. Thus, if there are two entities, one comes first and the other comes
  second; this order should not be switched.

  A \textbf{proper relation} is a type of relation in which given an $x$, you
  get only and exactly one $y$.
\end{definition}

But a function is not just any relation but only a \textit{proper relation}.
This means that, although all functions are relations because functions pair
two sets of information, not all relations are functions. This means that
functions are sub classification of relations.

\begin{example}
  As opposed to the previous case, let's flip things around now and assume that
  the domain is the set of everybody's birthdays. Let's imagine now that all
  family and friends have gathered for some event and someone ordered pizza,
  and she tells him only the birthday of the recipient. Too whom does the pizza
  guy deliver it to? What if nobody has the given birthday, or multiple people
  have a given birthday.
\end{example}

This means that the relation of "birthday indicates name" is not a proper
relation, which implies that it isn't a function. Given the relationship $(x,
y) = (\textrm{name}, \textrm{birthday})$, if there are four people with the
same birthday, there will be four different possibilities for $y =$ birthday.
For a relation to be a function, there must be only and exactly one  $y$ that
corresponds to a given $x$.

% subsection the_basics_of_functions (end)

\subsection{Inverse Functions}
\label{sub_sec:inverse_functions}

Recall that you can measure temperature using two scales: the Fahrenheit and the Celsius scales. The relationship between measurements in these two scales is as follows:

\begin{align*}
  F^{\circ} &= \frac{9}{5}C^{\circ} + 32 \\
  C^{\circ} &= \frac{5}{9}(F^{\circ} - 32).
\end{align*}

The first formula gives us a temperature in $F^{\circ}$ as a function of
$C^{\circ}$, and the second one gives us a temperature in $C^{\circ}$ as a
function of $F^{\circ}$. Because each formula undoes what the other one does,
they are examples of \textit{inverse functions}.

\begin{definition}[Inverse Functions]
  \label{def:inverse_functions}

  Two functions $f$ and $g$ are called \textbf{inverse functions} if the
  following statement is true:

  \begin{itemize}
    \item $g(f(x)) = x$ for all $x$ in the domain of $f$
    \item $f(g(x)) = x$ for all $x$ in the domain of $g$
  \end{itemize}

  An inverse function is denoted by $f^{-1}$.
\end{definition}

The inverse of a function has all the same points as the original function, but
the $x$'s and $y$'s are reversed.

\begin{example}
  If the original function has points:

  \[ \{ (2,3), (4,1), (6,-2) \} . \]

  Then the inverse function has points:

  \[ \{ (3,2), (1,4), (-2,6) \} . \]
\end{example}

In general, the graph of the inverse function $f^{-1}$ can be obtained from the
graph of $f$ by changing every point $(x,y)$ to the point $(y,x)$. This means
that the graph of $f^{-1}$ is the reflection of the graph $f$ in the line $y =
x$.

\begin{figure}[ht]
  \centering
  \incfig[1]{graph_of_inverse_function}
  \caption{Graph of Inverse Function}
  \label{fig:graph_of_inverse_function}
\end{figure}

The horizontal line test is used to determine whether a function has an inverse
that is also a function.

\begin{definition}[Horizontal Line Test]
  \label{def:horizontal_line_test}

  If no horizontal line intersects a graph of a given function in more than one
  point, then this function has an inverse. This is called the
  \textbf{horizontal line test}.
\end{definition}

For some functions, it is easy to find their inverses using basic algebraic
operations.

\begin{example}
  \label{exm:inverse_functions}

  Let's find the inverse function of the following function:

  \begin{align*}
    f(x) &= \frac{x - 1}{2} \\
    y &= \frac{x - 1}{2} \\
    x &= \frac{y - 1}{2} \\
    2x &= y - 1 \\
    y &= 2x + 1
  .\end{align*}
\end{example}

% subsection inverse_functions (end)

% section what_are_functions (end)

\section{Complex Numbers}
\label{sec:complex_numbers}

\begin{definition}[Complex Numbers]
  \label{def:complex_numbers}

  \textbf{Complex numbers} are numbers in the form $a + bi$, where $a$ and $b$ 
  are real numbers, and $i$ is the imaginary unit. $a$ is the real part of such
  number and $bi$ is the imaginary part of a complex number.
\end{definition}

\begin{definition}[Imaginary Number]
  \label{def:imaginary_number}

  An \textbf{imaginary number} is a number with square that is negative.
  Imaginary numbers have the form $bi$, where $b$ is a nonzero real number and
  $i$ is the imaginary unit, defined such that $i^{2} = -1$.
\end{definition}

\subsection{Operations with Complex Numbers}
\label{sub_sec:operations_with_complex_numbers}

You can add or multiply two complex numbers by treating $i$ as if it were a
variable and using algebraic distribute laws:

\begin{equation*}
  \begin{alignedat}{3}
   &\textrm{Sample $1$:} \qquad (5 + 41) + (9 - 2i) &&= 14 + 2i \\
   &\textrm{Sample $2$:} \qquad (2 - 3i)(5 + 4i) &&= 10 + 8i - 15i - 12i^{2} \\
   &                             &&= 10 - 7i + 12 \\
   &                             &&= 22 - 7i
  \end{alignedat}
.\end{equation*}

To perform complex numbers division, we must first be introduced to
\textit{complex conjugates}.

\begin{definition}[Complex Conjugtes]
  \label{def:complex_conjugtes}

  The complex number $a + bi$ and $a - bi$ are called \textbf{complex
  conjugates}. Their sum is a real number, and their product is a non-negative
  real number. The conjugate of the complex number $z = a + bi$ is denoted by
  $\bar{z} = a - bi$.
\end{definition}

\begin{note}
  When dividing complex numbers, multiply the numerator and the denominator by
  the conjugate of the denominator.
\end{note}

\begin{align*}
  \frac{3 + 4i}{2 - 3i} &= \frac{(3 + 4i)(2 + 3i)}{(2 - 3i)(2 + 3i)} \\
                        &= \frac{6 + 9i + 8i + 12i^{2}}{4 + 6i - 6i - 9i^{2}} \\
                        &= \frac{6 + 17i - 12}{4 + 9} = \frac{-6 + 17i}{13} = -\frac{6}{13} + \frac{17}{13}i
.\end{align*}

% subsection operations_with_complex_numbers (end)

% section complex_numbers (end)

% chapter algebra_tools_needed_to_study_trigonometry (end)

\newpage
